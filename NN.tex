%%%%%%%%%%%%%%%%%%%%%%%%%%%%%%%%%%%%%%%%%
% Stylish Article
% LaTeX Template
% Version 2.0 (13/4/14)
%
% This template has been downloaded from:
% http://www.LaTeXTemplates.com
%
% Original author:
% Mathias Legrand (legrand.mathias@gmail.com)
%
% License:
% CC BY-NC-SA 3.0 (http://creativecommons.org/licenses/by-nc-sa/3.0/)
%
%%%%%%%%%%%%%%%%%%%%%%%%%%%%%%%%%%%%%%%%%

%----------------------------------------------------------------------------------------
%	PACKAGES AND OTHER DOCUMENT CONFIGURATIONS
%----------------------------------------------------------------------------------------

\documentclass[fleqn,10pt]{SelfArx} % Document font size and equations flushed left

\usepackage{lipsum} % Required to insert dummy text. To be removed otherwise
\usepackage{datetime}

%----------------------------------------------------------------------------------------
%	COLUMNS
%----------------------------------------------------------------------------------------

\setlength{\columnsep}{0.55cm} % Distance between the two columns of text
\setlength{\fboxrule}{0.75pt} % Width of the border around the abstract

%----------------------------------------------------------------------------------------
%	COLORS
%----------------------------------------------------------------------------------------

\definecolor{color1}{RGB}{0,0,90} % Color of the article title and sections
\definecolor{color2}{RGB}{0,20,20} % Color of the boxes behind the abstract and headings

%----------------------------------------------------------------------------------------
%	HYPERLINKS
%----------------------------------------------------------------------------------------

\usepackage{hyperref} % Required for hyperlinks
\hypersetup{hidelinks,colorlinks,breaklinks=true,urlcolor=color2,citecolor=color1,linkcolor=color1,bookmarksopen=false,pdftitle={Title},pdfauthor={Author}}

%----------------------------------------------------------------------------------------
%	ARTICLE INFORMATION
%----------------------------------------------------------------------------------------

\JournalInfo{Zusammenfassung Neuronale Netze SS 2015} % Journal information
\CurrentStatus{Stand: \today, \currenttime Uhr} % Date of last modification
\Archive{} % Additional notes (e.g. copyright, DOI, review/research article)

\PaperTitle{Neuronale Netze} % Article title

\Authors{Benjamin Rupp\textsuperscript{1}*, Sophie von Schmettow\textsuperscript{2}} % Authors
%\affiliation{\textsuperscript{1}\textit{Department of Biology, University of Examples, London, United Kingdom}} % Author affiliation
%\affiliation{\textsuperscript{2}\textit{Department of Chemistry, University of Examples, London, United Kingdom}} % Author affiliation
%\affiliation{*\textbf{Corresponding author}: john@smith.com} % Corresponding author

\Keywords{Keyword1 --- Keyword2 --- Keyword3} % Keywords - if you don't want any simply remove all the text between the curly brackets
\newcommand{\keywordname}{Keywords} % Defines the keywords heading name

%----------------------------------------------------------------------------------------
%	ABSTRACT
%----------------------------------------------------------------------------------------

\Abstract{...}

%----------------------------------------------------------------------------------------

\begin{document}

\flushbottom % Makes all text pages the same height

\maketitle % Print the title and abstract box

\tableofcontents % Print the contents section

\thispagestyle{empty} % Removes page numbering from the first page
%----------------------------------------------------------------------------------------
%	UTILS
%----------------------------------------------------------------------------------------

%\begin{figure*}[ht]\centering % Using \begin{figure*} makes the figure take up the entire width of the page
%\includegraphics[width=\linewidth]{view}
%\caption{Wide Picture}
%\label{fig:view}
%\end{figure*}



%\begin{equation}
%\cos^3 \theta =\frac{1}{4}\cos\theta+\frac{3}{4}\cos 3\theta
%\label{eq:refname2}
%\end{equation}



%\begin{enumerate}[noitemsep] % [noitemsep] removes whitespace between the items for a compact look
%\item First item in a list
%\item Second item in a list
%\item Third item in a list
%\end{enumerate}



%\begin{figure}[ht]\centering
%\includegraphics[width=\linewidth]{results}
%\caption{In-text Picture}
%\label{fig:results}
%\end{figure}

%Reference to Figure \ref{fig:results}.


%\begin{table}[hbt]
%\caption{Table of Grades}
%\centering
%\begin{tabular}{llr}
%\toprule
%\multicolumn{2}{c}{Name} \\
%\cmidrule(r){1-2}
%First name & Last Name & Grade \\
%\midrule
%John & Doe & $7.5$ \\
%Richard & Miles & $2$ \\
%\bottomrule
%\end{tabular}
%\label{tab:label}
%\end{table}



%\begin{description}
%\item[Word] Definition
%\item[Concept] Explanation
%\item[Idea] Text
%\end{description}

%----------------------------------------------------------------------------------------
%	ARTICLE CONTENTS
%----------------------------------------------------------------------------------------

% http://www.blickinsbuch.de/3486243500&account=4907031511
\section{Einführung} 

% Motivation
% Neurobiologische Grundlagen
% Geschichte
% Terminologie, Biologisches Neuron hier
% Bestandteilte neuronaler Netze
% Konzepte des Konnektionusmus/Lernregeln
% Komponenten neuronaler Modelle
% Was es so für neuronale Modelle gibt
% bis einschließlich Kap. 6 von Zell

%------------------------------------------------
\section{Perzeptron}
%Kap. 7 bei Zell

\section{Backpropagation}
%Kap. 8 und 9 bei Zell

\section{Jordan \& Elman Netze}
% Kap. 11

\section{Gradientenverfahren für rekurrente Netze}
% 12

\section{Cascade-Correlation}
% 13

\section{Lernende Vektorquantisierung LVQ}
% 14

\section{Selbstorganisierende Karten (SOM)}
% 15

\section{Hopfield-Netze}
% 17

\section{Boltzmann-Maschine}
% 18

\section{Time-Delay-Netze}
% 24

\section{Minimierung von NN}
% 25

\section{Ausblick/Anwendung neuronaler Netze}
% 40
%------------------------------------------------


% Einteilung in den Folien:
% \section{Klassifikation/Mustererkennung}
% \section{Lernende Vektorquantisierung (LVQ) \& Verwandte Techniken}
%\section{Das Perzeptron (+ Biological Neurons)}
%\section{Features}
%\section{Backpropagation}
%\section{Learning Features: Autoencoders and Bottleneck Features}
%\section{Deep Learning}
%\section{Training Neural Networks with Reinforcement Learning}
%\section{Hopfield Nets and Boltzmann Machines}
%\section{Recurrent Neural Networks}
%\section{Deep Learning in Computer Vision}




%----------------------------------------------------------------------------------------
%	REFERENCE LIST
%----------------------------------------------------------------------------------------
\phantomsection
\bibliographystyle{unsrt}
\bibliography{sample}

%----------------------------------------------------------------------------------------

\end{document}